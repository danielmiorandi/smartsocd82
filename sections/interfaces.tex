In this section we will list the components integrated at the moment and the API subset used.

\subsection{Peer Manager}
The Peer Manager (PM) API specifications can be found at \url{http://demos.disi.unitn.it:8080/smartsociety/}. The PM is used for:
\begin{itemize}
\item {\bf Retrieving peers satisfying some requirements}: This request is made by the Orchestration Manager. Endpoint used:\\
	\url{???}
\item {\bf Creating collective}: This request is made by the Application/Application Runtime at the end of the negotiation. Endpoint used:\\
	\url{/peers/collectivePeer} (\textsc{POST})
\item {\bf Retrieving peers in a collective}: This request is made by the smartcom application, so that it will be able to retrieve peer information later. Endpoint used:\\
	\url{/peers/collectivePeer ???} (\textsc{GET})
\item {\bf Retrieving peers info}: This request is made by Smartcom application, this is required because Smartcom requires the information about the communication channel to be used. Endpoint used:\\
	\url{/peers/collectivePeer ???} (\textsc{GET})	
\end{itemize}


\subsection{Communication Middleware}
The SmartCom Communication Middleware APIs specifications can be found at 
url{https://github.com/tuwiendsg/SmartCom} and are further described in Deliverable D7.1. The Communication Middleware is mainly used for:
\begin{itemize}
\item {\bf Application to peer communication}: The Smart Society application sends messages to the peers belonging to a collective through Smartcom, that will deliver it using the preferred communication channel declared in the peer profile. Endpoint used: {at.ac.tuwien.dsg.smartcom.Communication}
\item {\bf Peer to application reply communication}: Peers can reply to messages from the application. The platform provides an endpoint to which the peers can send a reply to a given message. The message is identified by the conversation id, this attribute is also used to know which application
originally created the message, so that the platform can correctly dispatch it. 
\end{itemize}

\subsection{Orchestration Manager}
The Orchestration and Negotiation Manager APIs specification can be found at \url{https://bitbucket.org/rovatsos/smartsociety-internal/wiki/PeerManager/APIBGU} and are further described in Deliverable D6.2. The OCM is mainly used for:
\begin{itemize}
\item {\bf Posting a new task request}: In order to post a task request the following endpoint is used:\\
	\url{/applications/:app/taskRequests} (\textsc{POST})
\item {\bf Compose collectives adequate for fulfilling a given task:} this is carried out by the composition manager (see D6.2) and is executed every time a task request is posted in the system. Endpoint used:
  \url{/applications/:app/compositions} (\textsc{POST})
\item {\bf Negotiate with peers to create an agreed plan:} it negotiates with peers in order to have their explicit agreement for carrying out a given task. It returns a plan. Endpoint used:
  \url{/applications/:app/negotiations} (\textsc{POST})
  
\end{itemize}
