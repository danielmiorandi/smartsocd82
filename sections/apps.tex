The SmartSociety platform is meant to support a rather wide range of
social computation patterns (or templates). In order to provide
insight into the flexibility of the platform and the actual
interworking of components, we have developed sequence diagrams for two
'extreme' applications:
\begin{itemize}
\item AskSmartSociety! is a Q\&A service supporting hybridity. The
computational pattern here is that typical of micro-tasking
applications (\'a la Mechanical Turk, roughly speaking), where the
task in this corresponds to a question to be answered. The service
supports hybrid computation in that questions can be transparently
provided by machine peers or human peers. Quality criteria can be
specified in order to define when a chosen answer has to be presented
to the user. The scenario was used as the reference benchmark for the integration work carried out in WP8.

\item SmartShare is a ridesharing system able to account for user's
preferences and to compute recommendations based on the feedback
provided by other service users. It is what we call a 'full
negotiation' scenario, in which the computational task of finding an
agreement on the rides is left to individuals and collectives. The
platform in this case is used to carry out administration tasks, in
particular keeping track of the rides and ride requests, their status
and to maintain reputation of drivers and passengers.
\end{itemize}
In the following we present details about the two aforementioned
applications. 


\subsection{Example: Ask SmartSociety}\label{sec:asksmartsoc}
\textit{Ask SmartSociety} is a simple Questions and Answer service enabled by the SmartSociety platform which has been used as the benchamark for implementing the initial version of the SmartSociety platform. It focus on tourism, which is the reference domain to be used for validating the SmartSociety vision.\\
Ask SmartSociety! will be a service where users can post questions in natural languages and peers can provide answers. Peers providing answers can be humans (individuals or collectives) as well as machines (intelligent software agents). Peers can compose (forming collectives, hybrid or not) to provide answers. Answers can be ranked based on the reputation of the peers or on community ranking (similarly to stackoverflow). In some instances the user issuing the question can select an answer and provide feedback on the peer providing it.
Two examples (grounded in the tourism application scenarios) can help in understanding the features of the Ask SmartSociety! service:

\textit{Next week Peter will fly to Venice. He will be busy in meetings during the day but wants to explore some ‘hidden’ places at night. He could well explore various online tourist sites but he prefers to ask experts and local people. He could also google for relevant content, but he does not actually need an answer right away, he just needs to get it in one week. And having one system which allows him to query local experts, web-based recommendation services and incoming tourism institutions looks definitely appealing to him!
Alice is visiting Milano during the next week. It is his first time in Milan, and she is looking for a restaurant in the city centre. Since it is spring time, she would love eat outside and therefore  find a restaurant with a garden. Alice is also celiac, and she needs to find restaurants, which do have gluten free menus. She relies on the Ask SmartSociety! application for retrieving some suggestions on where to have dinner during her stay in Milan. She is looking for unconventional recommendations.}

Fig.~\ref{fig:dynamic_ask} illustrates how SmartSociety dynamic view is instantiated in the case of the \textit{AskSmartSociety!} application.

\begin{figure}[!hbt]
\centering
\includegraphics[width=0.9\textwidth]{./figs/ask/ask_dynamic}
\caption{Sequence diagram for \textit{AskSmartSociety!} application.}
\label{fig:dynamic_ask}
\end{figure}

Compared to the overall sequence diagram of Fig.~\ref{fig:dynamic_view}, this is a rather simplified use case where the user application consists of a mobile application, where users can post question. The question is dispatched to the corresponding application hosted in the runtime. Such application transforms the question into a computation task to be executed by peers. Such computation task is handled by the Orchestration Manager, which identifies the most suitable composition for completing the task. As an example, if the question regards restaurants, the Orchestration Manager might decide to use machine peers such as, e.g., TripAdvisor or Yelp, in combination with some \textit{local knowledge} provided by human peers living in the region where the request was created. Once the composition has been created, the application recruits the peers for the execution of the task. This could potentially involve the use of incentives in order to motivate peers participation.\\
After peers have been signed up for the given task, the task execution starts. In this case, the question is send to the peer over the preferred communication channel. In the case of machine peer, this corresponds to the proper APIs, or in the case of human peers cold be a mobile application. In the implemented scenario, we used a Twitter peer which used a hashtag to collect recommendations and a mobile application, through which users could receive questions and provide answers. In this case, the execution Manager is not used, as the application can interact directly with peers, exploiting the SmartCom functionalities.\\
Once the task is completed, and the answers are collected, the outcome of the computation is delivered to the users issuing the question.

Fig.~\ref{fig:askpeer} illustrates the US design that has been conducted for the mobile application.


\begin{figure}[!bht]
    \subfloat[Question form\label{subfig-1:dummy}]{%
      \includegraphics[width=0.2\textwidth]{./figs/ask/ask0.pdf}
    }
    \hfill
    \subfloat[Questions list\label{subfig-2:dummy}]{%
      \includegraphics[width=0.2\textwidth]{./figs/ask/ask1.pdf}
    }
	\subfloat[Question\label{subfig-2:dummy}]{%
      \includegraphics[width=0.2\textwidth]{./figs/ask/ask2.pdf}
    }
    	\subfloat[Quesitons status\label{subfig-2:dummy}]{%
      \includegraphics[width=0.2\textwidth]{./figs/ask/ask3.pdf}
    }
    	\subfloat[Question details\label{subfig-2:dummy}]{%
      \includegraphics[width=0.2\textwidth]{./figs/ask/ask4.pdf}
    }
    \caption{Ask SmartSociety UX design}
    \label{fig:askpeer}
\end{figure}


\subsubsection{Example: SmartShare}

TO BE ADDED
%\begin{figure}
%\centering
%\includegraphics[width=0.9\textwidth]{./figs/sequenceRide}
%\caption{Sequence diagram of the SmartShare application.}
%\label{fig:dynamic_share}
%\end{figure}
