In this deliverable we have described the core structure of the SmartSociety platform. At the moment the platform integrates three key components: peer manager, orchestration manager and communication middleware. This 'minified' version of the platform is sufficient for developing applications and run experiments. It further represents the core around which the other components will be integrated and deploy in order to realise the full-fledged version of the platform. 

In line with the DoW, the actual internal release of the first version of the platform (including user modules) represents MS18 and is due at M27. By M27 the integraiton roadmap foresees the integration of the provenance store and of the reputation manager. It will further include a first version of the execution manager and of the monitoring and analysis service. Security and privacy aspects will be integrated during the third year of the project, in particular within the scope of WP4 (where all personal information is stored). The programming framework development will be strictly aligned with the platform functionality enhancement and will support the easy registration and deployment of applications. The context manager will be integrated with the peer manager during year-3. The incentives manager will be integrated in a later stage of the project, the actual integration date depending on the progress of research activities within WP5. Further work will be carried out in a dialogue with WP1 in terms of the governance of the platform. Last, but not least, the development of the platform will also reflect the work on exploitation plans and potential business models carried out within the scope of WP10.

The second platform prototype (including validation results) will be delivered as D8.3 at M30. 

% In this deliverable we have introduced:
% \begin{itemize}
% \item An analysis of requirements for the SmartSociety platform, elicited by working in  cooperation with the other work packages and covering both functional and non-functional properties. 
% \item A system-level architecture, encompassing:
% \begin{itemize}
% \item the definition of the logical and functional role of each platform component;
% \item a definition of the interactions among components;
% \item the preliminary identification of modules within each component, in line with the progress of single technical WPs.
% \end{itemize}
% \end{itemize}

% This document represents the starting point for the integration activities taking place in the second year of the project (T8.2 and T8.3). The overall architecture will be iteratively revised, improved and extended during the second year of the project, leading to the delivery, at M24, of a first prototypical implementation of the platform. D8.1 will
% serve as reference document and handbook for the development and prototyping activities to be carried out within WP2-WP7, defining clearly the role of the respective components in the overall platform architecture and the interaction patterns with other components. The architecture will be maintained as a living reference document; the release of the first prototype at M24 will go hand in hand with the provisioning of a revised architectural specification, accounting for the evolution of perspectives during year-2.

% Last, it is worth remarking that WP8 played an important {\it integration}
% role in the first year of activities of the SmartSociety project. In order to reach
% a consistent view on the architecture of the SmartSociety platform, indeed,
% WP8 established and fostered an active and open discussion among the
% technical WPs (WP2-WP7). The agreement reached among WPs in terms of logical role, functionalities of their components, and interaction patterns represents an important contribution of WP8 in year-1. 


