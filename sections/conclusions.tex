Summarizes the content of the deliverable and describes the integration steps that will be carried out in Y-3 (integration roadmap).

% In this deliverable we have introduced:
% \begin{itemize}
% \item An analysis of requirements for the SmartSociety platform, elicited by working in  cooperation with the other work packages and covering both functional and non-functional properties. 
% \item A system-level architecture, encompassing:
% \begin{itemize}
% \item the definition of the logical and functional role of each platform component;
% \item a definition of the interactions among components;
% \item the preliminary identification of modules within each component, in line with the progress of single technical WPs.
% \end{itemize}
% \end{itemize}

% This document represents the starting point for the integration activities taking place in the second year of the project (T8.2 and T8.3). The overall architecture will be iteratively revised, improved and extended during the second year of the project, leading to the delivery, at M24, of a first prototypical implementation of the platform. D8.1 will
% serve as reference document and handbook for the development and prototyping activities to be carried out within WP2-WP7, defining clearly the role of the respective components in the overall platform architecture and the interaction patterns with other components. The architecture will be maintained as a living reference document; the release of the first prototype at M24 will go hand in hand with the provisioning of a revised architectural specification, accounting for the evolution of perspectives during year-2.

% Last, it is worth remarking that WP8 played an important {\it integration}
% role in the first year of activities of the SmartSociety project. In order to reach
% a consistent view on the architecture of the SmartSociety platform, indeed,
% WP8 established and fostered an active and open discussion among the
% technical WPs (WP2-WP7). The agreement reached among WPs in terms of logical role, functionalities of their components, and interaction patterns represents an important contribution of WP8 in year-1. 


