% This report represents the outcome of the activities carried out within task \textit{T8.1, Platform
% requirements analysis and design} during the first year of the project in terms of analysis of requirements of the SmartSociety platform and system design. 

% Requirements are organized around the many heterogeneous dimensions involved in the design of a complex socio-technical system such as SmartSociety. The requirements were elicited from (i) the analysis of the application scenarios performed in WP9 (ii) the interdisciplinary foundations worked out in WP1 (iii) discussion and coordination with WP2-WP7 in order to ensure project-wide alignment in terms of the outcomes of the technical WPs. 

% The document includes also a first, high-level architecture of the SmartSociety platform. The design has been inspired by four principles. First, the need to meet the requirements identified in the first part of the deliverable. Second, the need to clearly identify how the outcomes of the technical WPs (WP2-WP7) will be integrated in the platform prototype. Third, to ensure a sufficient level flexibility in the way the platform can be used in order to sustain the in-field activities to be carried out for validation purposes in WP9. Fourth, to promote the re-use of existing infrastructures, components and modules in the development process in order to meet the criteria of economy, efficiency and effectiveness.

% The deliverable includes the identification of a number of key components, together with a description of their logical and functional role. Modules of each component are preliminarily identified, coherently with the progress of the single WPs. Interfaces and interactions among sub-systems are described and analysed.

% Finally, the ride-sharing scenario used in the first year of the project has been mapped to the proposed architecture in order to provide an example, for internal Consortium purposes, of how applications could be deployed on the platform.

% The deliverable represents a first step towards the implementation of a first prototype version, which will be released at M24.
TBA
